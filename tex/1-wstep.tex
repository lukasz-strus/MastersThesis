\newpage
\section{Wstęp}
Dynamiczny rozwój technologii informatycznych oraz rosnąca złożoność systemów biznesowych sprawiają, że kluczowym wyzwaniem w tworzeniu oprogramowania staje się utrzymanie wysokiej jakości kodu oraz jego łatwej rozbudowy i konserwacji. W odpowiedzi na te potrzeby coraz większą popularność zdobywają podejścia architektoniczne, takie jak Domain-Driven Design (DDD) \cite{evans2004ddd} oraz Clean Architecture \cite{unclebob2018cleanarchitecture}, które promują wyraźny podział odpowiedzialności, silne modelowanie domeny oraz elastyczność w rozwijaniu systemów.

Implementacja aplikacji w oparciu o te paradygmaty wymaga jednak znacznej liczby powtarzalnych czynności – tworzenia struktur projektowych, definiowania podstawowych klas i interfejsów czy utrzymywania spójności warstw systemu. Proces ten, choć niezbędny, jest czasochłonny i podatny na błędy, zwłaszcza w początkowej fazie projektu.

Celem niniejszej pracy magisterskiej jest stworzenie narzędzia, które zautomatyzuje kluczowe elementy wdrażania DDD w technologii .NET, eliminując konieczność ręcznego tworzenia powtarzalnych struktur. Opracowany generator kodu oparty na Roslyn \cite{microsoft2024roslyn} umożliwi automatyczne generowanie klas domenowych, a także szkieletów aplikacji zgodnych z zasadami Clean Architecture \cite{unclebob2018cleanarchitecture}.

Dodatkowo, w ramach pracy powstała biblioteka fundamentalnych komponentów DDD (m.in. Entity, ValueObject, AggregateRoot) \cite{evans2004ddd}, które stanowią solidną podstawę do tworzenia nowych systemów i mogą być wykorzystywane w wielu projektach. Dzięki temu rozwiązaniu programiści otrzymują spójne i powtarzalne narzędzia, wspierające zarówno jakość kodu, jak i efektywność pracy zespołu.

W dalszych rozdziałach pracy przedstawiono analizę istniejących narzędzi do generowania kodu w .NET, omówiono założenia projektowe oraz opisano proces implementacji generatora i biblioteki. Praca kończy się prezentacją wyników, oceną użyteczności rozwiązania oraz propozycjami dalszego rozwoju narzędzia.